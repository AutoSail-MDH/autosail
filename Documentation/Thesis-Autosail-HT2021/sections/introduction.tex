\section{Inledning (Introduction)}
\label{sec:intro}

This is an example to illustrate the use of the Latex template to write the thesis report. The text for the sections and subsections should be adated to best reflect the content of each section. For example, the "problem formulation" does not need to be a subsection of the introduction. It can be a complete
section on its own (if there is enough material). 

Inledning kan ses som en expanderad version av sammanfattningen. Du kan ha ungefär samma struktur men med ett eller två stycken för varje punkt i sammanfattningen. Följande bör ingå:
\begin{itemize}
\item[--]	Presentation av området och ämnet för arbetet. Detta bör komma tidigt och ska fånga intresset. Här kan ingå kort om bakgrund och eventuellt viktiga definitioner av begrepp
\item[--]	Du kan kort beskriva tänkt målgrupp för rapporten, vilka har du skrivit för?
\item[--]	Kort översikt över tidigare arbeten och dessas begränsningar 
\item[--]	Presentation av uppgiften inklusive syfte och frågeställning
\item[--]	Beskrivning av hur du angripit uppgiften, metod och varför denna är lämplig
\item[--]	Motivation: varför uppgiften är intressant, vilka de relevanta frågeställningarna är, varför ditt angreppssätt är bra och varför resultaten är viktiga.
\item[--]	Beskrivning av de viktigaste resultaten och dessas begränsningar samt vad som är nytt i ditt arbete
\item[--]	översikt av rapporten
\end{itemize}

Du kan diskutera betydelsen av slutsatserna men inledningen ska bara innehålla kort summering av resultaten. Ingen specialiserad terminologi eller matematik bör vara med här. 

Inledningen kan skrivas som en tratt: område – delområde – uppgift – eventuell deluppgift– syfte. Du leder då läsaren mot en gradvis mer detaljerad och specifik förståelse av uppgiften och syftet. I slutet av inledningen ska läsaren och du ha en bas av gemensam förståelse. Läsaren ska förstå uppgiften, arbetets omfattning, metod och dess viktigaste bidrag, dvs vad som är nytt i ditt arbete. 

även de andra avsnitten i rapporten kan behöva en kort inledning i början, för att läsaren ska förstå syftet med varje avsnitt och dess plats i rapporten.


\subsection{Problemformulering (Problem Formulation)} 

I detta avsnitt formulerar och preciserar du de tre viktiga sakerna syfte, frågeställning och motivation. Du ska här presentera uppgiften på ett tydligt sätt, både på hög nivå och i detalj, samt diskutera varför den är viktig. Redogör för antaganden och begränsningar. Från beskrivningen av uppgiften kan du sedan formulera syftet och frågeställningen. Tänk på att när syftet är uppfyllt så ska frågeställningen kunna besvaras. Det är också viktigt att syfte och motivation hänger ihop. När syftet och frågeställningen är klara kan du börja utveckla målen, målen ska uppnås för att nå syftet. Varje mål ska vara litet, genomförbart och möjligt att utvärdera.  


\emph{Tips!} Skriv ner din forskningsfråga på en lapp som du sätter vid skärmen. På så sätt får du hjälp att alltid hålla forskningsfrågan i åtanke när du arbetar med rapporten.

This is how to use the references~\cite{Berndtsson607210, Blomkvist2014} or~\cite{Turing1950}.



